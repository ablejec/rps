% -*- TeX:Rnw -*-
% ----------------------------------------------------------------
% .R Sweave file  ************************************************
% ----------------------------------------------------------------
%%
% \VignetteIndexEntry{}
% \VignetteDepends{}
% \VignettePackage{}
%\documentclass[a4paper,12pt]{article}
\usepackage[slovene]{babel}
\usepackage[utf8]{inputenc} %% must be here for Sweave encoding check
\newcommand{\SVNRevision}{$ $Rev: 3 $ $}
%\newcommand{\SVNDate}{$ $Date:: 2009-02-2#$ $}
\newcommand{\SVNId}{$ $Id: program.Rnw 3 2009-02-22 17:36:08Z ABlejec $ $}
%\usepackage{babel}
%\input{abpkgB}
%\input{abpkg}
\input{abBeam}
\input{abcmd}
%\input{abpage}
\usepackage{pgf,pgfarrows,pgfnodes,pgfautomata,pgfheaps,pgfshade}
\usepackage{amsmath,amssymb}
\usepackage{colortbl}
\usepackage{Sweave}
\usepackage{animate}
\input{mysweaveB}
\newcommand{\BV}{}
\newcommand{\EV}{}
\newcommand{\myemph}[1]{{\color{Sgreen} \textit{#1}}}
\input{./figs/SwPres-concordance}
%\SweaveOpts{echo=false}
%\usepackage{lmodern}
%\input{abfont}

% ----------------------------------------------------------------
\title{Drobtinice\footnote{... kot bi temu rekel kolega Batagelj :)}}
\author{A. Blejec}
%\address{}%
%\email{}%
%
%\thanks{}%
%\subjclass{}%
%\keywords{}%

%\date{}%
%\dedicatory{}%
%\commby{}%
\begin{document}
\mode<article> {\maketitle}
\mode<presentation> {\frame{\titlepage}}
\tableofcontents
% ----------------------------------------------------------------
\begin{abstract}
 Razna vprašanja in namigi
\end{abstract}
% -------------------------------------------------------------
%% Sweave settings for includegraphics default plot size (Sweave default is 0.8)
%% notice this must be after begin{document}
% \setkeys{Gin}{width=0.9\textwidth}
\setkeys{Gin}{width=0.7\textwidth}
% ----------------------------------------------------------------
%% |--------------------->>>>>>
\begin{frame}[fragile]
\frametitle{Nekaj tem s konca prejšnjih predavanj}
\begin{itemize}
   \item I/O
   \item kontrolne strukture
   \item paketi !admin
   \item ML
   \item linearni model
 \end{itemize}
\end{frame}
%% <<<<<<---------------------|
\section{I/O}
%% |--------------------->>>>>>
\begin{frame}[fragile]
\frametitle{I/O }
\begin{itemize}
  \item R je glede branja in pisanja precej fleksibilen.
  \item Najbolj zanesljive so navadne text datoteke ... \emph{itak}
  \item Več o tem je na \url{http://ablejec.nib.si/R/HowTo-IOS.pdf}
\end{itemize}

\end{frame}
%% <<<<<<---------------------|
\section{kontrolne strukture}
%% |--------------------->>>>>>
\begin{frame}[fragile]
\frametitle{Kontrolne strukture}
Kontrolne strukture omogočajo odločitve (\code{if})\\%
in ponavljanje izvedbe ukazov (\code{for, while, repeat})
\begin{Schunk}
\begin{Sinput}
 if (interactive()) help("if")
\end{Sinput}
\end{Schunk}
\end{frame}
%% <<<<<<---------------------|

%% |--------------------->>>>>>
\begin{frame}[fragile]
\frametitle{\code{if}}
\code{if(cond) expr}\\
\code{if(cond) cons.expr  else  alt.expr}
\begin{Schunk}
\begin{Sinput}
 x <- 3
 meja <- 5
 if (x < meja) cat(x, "je manj od", meja, "\n")
\end{Sinput}
\begin{Soutput}
3 je manj od 5 
\end{Soutput}
\begin{Sinput}
 if (x < meja) cat(x, "je manj od", meja, "\n") else cat("Večje\n")
\end{Sinput}
\begin{Soutput}
3 je manj od 5 
\end{Soutput}
\end{Schunk}
Spremenite mejo v \code{7} in poskusite znova.
\end{frame}
%% <<<<<<---------------------|

%% |--------------------->>>>>>
\begin{frame}[fragile]
\frametitle{Več kot en ukaz: \verb"{ }"}
Več ukazov, ki sodijo skupaj zapremo z \verb"{ }" v blok

\begin{Schunk}
\begin{Sinput}
 x <- 3
 meja <- 5
 if (x < meja) {
     lbl <- "Vsota: "
     y <- x + meja
 } else {
     lbl <- "Produkt: "
     y <- x * meja
 }
 cat(lbl, y, "\n")
\end{Sinput}
\begin{Soutput}
Vsota:  8 
\end{Soutput}
\end{Schunk}
\end{frame}
%% <<<<<<---------------------|

%% |--------------------->>>>>>
\begin{frame}[fragile]
\frametitle{\code{ifelse}}
\code{ifelse(test, yes, no)}
\begin{Schunk}
\begin{Sinput}
 x <- c(2:-2)
 ifelse(x >= 0, x, NA)
\end{Sinput}
\begin{Soutput}
[1]  2  1  0 NA NA
\end{Soutput}
\begin{Sinput}
 sqrt(ifelse(x >= 0, x, NA))
\end{Sinput}
\begin{Soutput}
[1] 1.414214 1.000000 0.000000       NA       NA
\end{Soutput}
\end{Schunk}
\end{frame}
%% <<<<<<---------------------|
%% |--------------------->>>>>>
\begin{frame}[fragile]
\frametitle{\code{switch}}
\begin{Schunk}
\begin{Sinput}
 centre <- function(x, type) {
     switch(type, mean = mean(x), median = median(x), 
         trimmed = mean(x, trim = 0.1))
 }
 x <- rcauchy(10)
 centre(x, "mean")
\end{Sinput}
\begin{Soutput}
[1] -1.532892
\end{Soutput}
\begin{Sinput}
 centre(x, "median")
\end{Sinput}
\begin{Soutput}
[1] -1.13698
\end{Soutput}
\begin{Sinput}
 centre(x, "trimmed")
\end{Sinput}
\begin{Soutput}
[1] -1.29566
\end{Soutput}
\end{Schunk}
\end{frame}
%% <<<<<<---------------------|

%% |--------------------->>>>>>
\begin{frame}[fragile]
\frametitle{Zanka \code{for}}
Uporabljamo, kadar je znano število ponavljanj\\ ali pa seznam vrednosti, za katere se stvar ponovi.\\
V \R ponavadi lahko nadomestimo z vektorskim načinom\\ ali pa s katero od \code{apply} funkcij. Glej paket \pkg{plyr}.
\begin{Schunk}
\begin{Sinput}
 for (x in 10:13) {
     y <- x^2
     print(y)
 }
\end{Sinput}
\begin{Soutput}
[1] 100
[1] 121
[1] 144
[1] 169
\end{Soutput}
\begin{Sinput}
 for (lbl in c("a", "e", "i", "o")) cat(paste("šal", 
     lbl, ", ", sep = ""))
\end{Sinput}
\begin{Soutput}
šala, šale, šali, šalo, 
\end{Soutput}
\end{Schunk}
\end{frame}
%% <<<<<<---------------------|

%% |--------------------->>>>>>
\begin{frame}[fragile]
\frametitle{Animacija\\Kam gre  porazdelitev $t$, ko naraščajo stopinje prostosti?}
\begin{Schunk}
\begin{Sinput}
 n <- seq(1, 30, 0.01)
 x <- seq(-5, 5, length = 100)
 for (df in n) {
     plot(x, dt(x, df), type = "l", lwd = 7, col = 4, 
         main = df, ylim = c(0, 0.4))
     lines(x, dnorm(x), type = "l", lwd = 3, col = 2)
 }
 df
\end{Sinput}
\end{Schunk}
\end{frame}
%% <<<<<<---------------------|

%% |--------------------->>>>>>
\begin{frame}[fragile]
\frametitle{Zanka se vrti do izpolnitve pogoja}
\code{while(cond) expr}\\
Število poskusov do prvega uspeha, če je $p=0.2$
\begin{Schunk}
\begin{Sinput}
 set.seed(123)
 (x <- as.numeric(runif(20) < 0.2))
\end{Sinput}
\begin{Soutput}
 [1] 0 0 0 0 0 1 0 0 0 0 0 0 0 0 1 0 0 1 0 0
\end{Soutput}
\begin{Sinput}
 i <- 1
 while (x[i] != 1) {
     i <- i + 1
 }
 i
\end{Sinput}
\begin{Soutput}
[1] 6
\end{Soutput}
\end{Schunk}

Bolj \R-jevsko:\vspace{-0.5cm}
\begin{Schunk}
\begin{Sinput}
 which(x == 1)
\end{Sinput}
\begin{Soutput}
[1]  6 15 18
\end{Soutput}
\begin{Sinput}
 which(x == 1)[1]
\end{Sinput}
\begin{Soutput}
[1] 6
\end{Soutput}
\end{Schunk}
\end{frame}
%% <<<<<<---------------------|

%% |--------------------->>>>>>
\begin{frame}[fragile]
\frametitle{\code{repeat}}
\begin{Schunk}
\begin{Sinput}
 x
\end{Sinput}
\begin{Soutput}
 [1] 0 0 0 0 0 1 0 0 0 0 0 0 0 0 1 0 0 1 0 0
\end{Soutput}
\begin{Sinput}
 i <- 1
 repeat {
     if (x[i] == 1) 
         break
     i <- i + 1
 }
 i
\end{Sinput}
\begin{Soutput}
[1] 6
\end{Soutput}
\end{Schunk}
\end{frame}
%% <<<<<<---------------------|

%% |--------------------->>>>>>
\begin{frame}[fragile]
\frametitle{Stopnja značilnosti}
\begin{Schunk}
\begin{Sinput}
 stars <- function(p) {
     alfa <- c(0.05, 0.01, 0.001, 0)
     st <- ""
     i <- 1
     while (p <= alfa[i]) {
         st <- paste(st, "*", sep = "")
         i <- i + 1
     }
     if (st == "") 
         st <- "NS"
     return(st)
 }
\end{Sinput}
\end{Schunk}

\begin{Schunk}
\begin{Sinput}
 p <- 0.002
 cat("p =", p, stars(p), "\n")
\end{Sinput}
\begin{Soutput}
p = 0.002 ** 
\end{Soutput}
\begin{Sinput}
 sapply(c(0.5, 0.05, 0.005, 5e-04), stars)
\end{Sinput}
\begin{Soutput}
[1] "NS"  "*"   "**"  "***"
\end{Soutput}
\end{Schunk}
\end{frame}
%% <<<<<<---------------------|

\section{paketi !admin}


%% |--------------------->>>>>>
\begin{frame}[fragile]
\frametitle{paketi !admin}
\fct{.libPaths}
\begin{Schunk}
\begin{Sinput}
 .libPaths()
\end{Sinput}
\begin{Soutput}
[1] "D:/RUSER/r/win-library/2.15"        
[2] "C:/Program Files/R/R-2.15.1/library"
\end{Soutput}
\end{Schunk}

Npr:

\code{.libPaths("H:/R/library")}

\end{frame}
%% <<<<<<---------------------|

\section{ML}
%% |--------------------->>>>>>
\begin{frame}[fragile]
\frametitle{ML - metoda največjega verjetja}
\url{http://ablejec.nib.si/pub/rps/MetodeS.pdf}
\end{frame}
%% <<<<<<---------------------|




%% |--------------------->>>>>>
\begin{frame}[fragile]
\frametitle{Metoda največjega verjetja (ML)}
\begin{center}
 \vspace{0pt} 
\animategraphics[scale=0.45,poster=first,every=1,controls]{25}{./figs/SwPres-mleMean}{}{} %
\end{center}
\end{frame}
%% <<<<<<---------------------|

%% |--------------------->>>>>>
\begin{frame}[fragile]
\frametitle{Mediana, aritmetična  in geometrijska sredina}
\url{http://bit.ly/XA8qdX}
\end{frame}
%% <<<<<<---------------------|


\section{linearni model}
%% |--------------------->>>>>>
\begin{frame}[fragile]
\frametitle{linearni model}

\end{frame}
%% <<<<<<---------------------|


% ----------------------------------------------------------------
\bibliographystyle{amsplain}
\bibliography{ab-general}
\end{document}
% ----------------------------------------------------------------
