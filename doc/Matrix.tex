% -*- TeX:Rnw -*-
% ----------------------------------------------------------------
% .R Sweave file  ************************************************
% ----------------------------------------------------------------
%%
% \VignetteIndexEntry{}
% \VignetteDepends{}
% \VignettePackage{}
%\documentclass[a4paper,12pt]{article}
\usepackage[slovene]{babel}
\usepackage[utf8]{inputenc} %% must be here for Sweave encoding check
\newcommand{\SVNRevision}{$ $Rev: 3 $ $}
%\newcommand{\SVNDate}{$ $Date:: 2009-02-2#$ $}
\newcommand{\SVNId}{$ $Id: program.Rnw 3 2009-02-22 17:36:08Z ABlejec $ $}
%\usepackage{babel}
%\input{abpkgB}
%\input{abpkg}
\input{abBeam}
\input{abcmd}
%\input{abpage}
\usepackage{pgf,pgfarrows,pgfnodes,pgfautomata,pgfheaps,pgfshade}
\usepackage{amsmath,amssymb}
\usepackage{colortbl}
\usepackage{Sweave}
\input{mysweaveB}
\newcommand{\BV}{}
\newcommand{\EV}{}
\newcommand{\myemph}[1]{{\color{Sgreen} \textit{#1}}}
\input{./figs/SwPres-concordance}
%\SweaveOpts{echo=false}
%\usepackage{lmodern}
%\input{abfont}

% ----------------------------------------------------------------
\title{Matrike}
\author{A. Blejec}
%\address{}%
%\email{}%
%
%\thanks{}%
%\subjclass{}%
%\keywords{}%

%\date{}%
%\dedicatory{}%
%\commby{}%
\begin{document}
\mode<article> {\maketitle}
\mode<presentation> {\frame{\titlepage}}
\tableofcontents
% ----------------------------------------------------------------
\begin{abstract}
 Nekaj o funkcijah za delo z matrikami in nekaj primerov iz multivariatne analize.
\end{abstract}
% -------------------------------------------------------------
%% Sweave settings for includegraphics default plot size (Sweave default is 0.8)
%% notice this must be after begin{document}
% \setkeys{Gin}{width=0.9\textwidth}
\setkeys{Gin}{width=0.7\textwidth}
% ----------------------------------------------------------------


Zbrali smo nekaj podatkov o študentih, s katerimi si bomo lahko poskusili odgovoriti.

%% <<<<<<---------------------|
Nato zberemo podatke, s katerimi bomo poskusili odgovoriti na vprašanja. Ker predvidevamo, da nas bo zanimalo še kaj, zberemo podatke o še nekaj spremenljivkah.
\begin{Schunk}
\begin{Sinput}
 lfn <- "Podatki2012.txt"
 fpath <- "http://bit.ly/16oBVpR"
\end{Sinput}
\end{Schunk}

%% |--------------------->>>>>>
\begin{frame}[fragile]
\frametitle{Podatki}
Podaki so o študentih 3. letnika biologije v letu 2012/13 so v datoteki \file{lfn} in na \url{http://bit.ly/16oBVpR}
\begin{Schunk}
\begin{Sinput}
 fpath <- file.path("../data", lfn)
 data <- read.table(fpath, header = TRUE, sep = "\t")
 names(data)
\end{Sinput}
\begin{Soutput}
 [1] "starost" "mesec"   "spol"    "masa"    "visina"  "roke"   
 [7] "cevelj"  "lasje"   "oci"     "mati"    "oce"     "majica" 
\end{Soutput}
\end{Schunk}
\end{frame}
%% <<<<<<---------------------|

%% |--------------------->>>>>>
\begin{frame}[fragile]
\frametitle{Popravljanje podatkov}
Odstranimo ta starga
\begin{Schunk}
\begin{Sinput}
 data <- data[data$starost < 30, ]
\end{Sinput}
\end{Schunk}
Podatke o mesecu 0 spremenimo v \code{NA}

\begin{Schunk}
\begin{Sinput}
 data[data$mesec == 0, "mesec"] <- NA
 table(data$mesec)
\end{Sinput}
\begin{Soutput}
 1  2  3  4  5  6  7  8  9 10 11 
 1  3  2  3  4  3  7  5  2  5  6 
\end{Soutput}
\end{Schunk}
\end{frame}
%% <<<<<<---------------------|

%% |--------------------->>>>>>
\begin{frame}[fragile]
\frametitle{Sprememba podatkov o velikosti majice}
\begin{Schunk}
\begin{Sinput}
 table(data$majica)
\end{Sinput}
\begin{Soutput}
 L  M  S XL XS 
 4 19 16  1  2 
\end{Soutput}
\begin{Sinput}
 data$majica[data$majica == "XL"] <- "L"
 data$majica[data$majica == "XS"] <- "S"
 data$majica <- ordered(data$majica, levels = c("S", 
+     "M", "L"))
 str(data$majica)
\end{Sinput}
\begin{Soutput}
 Ord.factor w/ 3 levels "S"<"M"<"L": 1 1 1 2 3 1 1 2 2 3 ...
\end{Soutput}
\begin{Sinput}
 table(data$majica)
\end{Sinput}
\begin{Soutput}
 S  M  L 
18 19  5 
\end{Soutput}
\end{Schunk}
\end{frame}
%% <<<<<<---------------------|


\section{Matrični račun}
\subsection{Priprava vektorjev in matrik}

%% |--------------------->>>>>>
\begin{frame}[fragile]
\frametitle{Vektorji}
Vektor pripravimo kot $n$-terico s pomočjo funkcije \texttt{c}:
\begin{Schunk}
\begin{Sinput}
 x <- c(1, 2, -1)
 y <- c(2, 1, 6)
 x
\end{Sinput}
\begin{Soutput}
[1]  1  2 -1
\end{Soutput}
\begin{Sinput}
 y
\end{Sinput}
\begin{Soutput}
[1] 2 1 6
\end{Soutput}
\end{Schunk}
Vektor ima dolžino:
\begin{Schunk}
\begin{Sinput}
 length(x)
\end{Sinput}
\begin{Soutput}
[1] 3
\end{Soutput}
\end{Schunk}
\end{frame}
%% <<<<<<---------------------|

%% |--------------------->>>>>>
\begin{frame}[fragile]
\frametitle{Vektorska aritmetika}
Z vektorji lahko računamo po komponentah:

\begin{Schunk}
\begin{Sinput}
 2 * x
\end{Sinput}
\begin{Soutput}
[1]  2  4 -2
\end{Soutput}
\begin{Sinput}
 x + y
\end{Sinput}
\begin{Soutput}
[1] 3 3 5
\end{Soutput}
\begin{Sinput}
 x - y
\end{Sinput}
\begin{Soutput}
[1] -1  1 -7
\end{Soutput}
\begin{Sinput}
 x * y
\end{Sinput}
\begin{Soutput}
[1]  2  2 -6
\end{Soutput}
\begin{Sinput}
 x/y
\end{Sinput}
\begin{Soutput}
[1]  0.5000000  2.0000000 -0.1666667
\end{Soutput}
\begin{Sinput}
 x^y
\end{Sinput}
\begin{Soutput}
[1] 1 2 1
\end{Soutput}
\end{Schunk}
\end{frame}
%% <<<<<<---------------------|


\subsection{Matrike}

%% |--------------------->>>>>>
\begin{frame}[fragile]
\frametitle{}
Za ročni vnos in urejanje matrik lahko uporabimo urejevalec tabel \code{edit}.

\begin{Schunk}
\begin{Sinput}
 if (interactive()) {
+     X0 <- make.matrix(n = 5, m = 2)
+     X0
+ }
\end{Sinput}
\end{Schunk}
\end{frame}
%% <<<<<<---------------------|

Na voljo imamo funkcije, za manipulacijo matrik in pomembne operacije nad matrikami
Matrike množimo s pomočjo operatorja \verb"%*%",
transpozicijo pa naredimo s pomočjo funkcije \code{t}. Običajni
aritmetični parametri delujejo  po komponentah:

%% |--------------------->>>>>>
\begin{frame}[fragile]
\frametitle{Množenje \code{\%*\%} in transpozicija \fct{t}}
\begin{Schunk}
\begin{Sinput}
 (B <- cbind(x, y))
\end{Sinput}
\begin{Soutput}
      x y
[1,]  1 2
[2,]  2 1
[3,] -1 6
\end{Soutput}
\begin{Sinput}
 (A <- matrix(c(1, 2, -1, 0), 2, 2))
\end{Sinput}
\begin{Soutput}
     [,1] [,2]
[1,]    1   -1
[2,]    2    0
\end{Soutput}
\begin{Sinput}
 (C <- (B %*% A))
\end{Sinput}
\begin{Soutput}
     [,1] [,2]
[1,]    5   -1
[2,]    4   -2
[3,]   11    1
\end{Soutput}
\begin{Sinput}
 t(C)
\end{Sinput}
\begin{Soutput}
     [,1] [,2] [,3]
[1,]    5    4   11
[2,]   -1   -2    1
\end{Soutput}
\begin{Sinput}
 B + C
\end{Sinput}
\begin{Soutput}
      x  y
[1,]  6  1
[2,]  6 -1
[3,] 10  7
\end{Soutput}
\end{Schunk}
\end{frame}
%% <<<<<<---------------------|

\subsection{Nekaj statističnih zanimivosti matrik}

%% |--------------------->>>>>>
\begin{frame}[fragile]
\frametitle{}
\begin{Schunk}
\begin{Sinput}
 X0 <- cbind(x, y)
 X0
\end{Sinput}
\begin{Soutput}
      x y
[1,]  1 2
[2,]  2 1
[3,] -1 6
\end{Soutput}
\begin{Sinput}
 X <- X0
 dim(X)
\end{Sinput}
\begin{Soutput}
[1] 3 2
\end{Soutput}
\begin{Sinput}
 n <- dim(X)[1]
 dimnames(X)
\end{Sinput}
\begin{Soutput}
[[1]]
NULL

[[2]]
[1] "x" "y"
\end{Soutput}
\end{Schunk}
\end{frame}
%% <<<<<<---------------------|



%% |--------------------->>>>>>
\begin{frame}[fragile]
\frametitle{Povprečja}
\begin{Schunk}
\begin{Sinput}
 M <- apply(X0, 2, mean)
 M
\end{Sinput}
\begin{Soutput}
        x         y 
0.6666667 3.0000000 
\end{Soutput}
\end{Schunk}
matrika povprečij
\begin{Schunk}
\begin{Sinput}
 t(t(rep(1, n))) %*% t(M)
\end{Sinput}
\begin{Soutput}
             x y
[1,] 0.6666667 3
[2,] 0.6666667 3
[3,] 0.6666667 3
\end{Soutput}
\end{Schunk}


\end{frame}
%% <<<<<<---------------------|

%% |--------------------->>>>>>
\begin{frame}[fragile]
\frametitle{}
vsredinjena matrika podatkov
\begin{Schunk}
\begin{Sinput}
 X <- scale(X, scale = F)
 X
\end{Sinput}
\begin{Soutput}
              x  y
[1,]  0.3333333 -1
[2,]  1.3333333 -2
[3,] -1.6666667  3
attr(,"scaled:center")
        x         y 
0.6666667 3.0000000 
\end{Soutput}
\end{Schunk}
vektor povprečij izvlečemo s funkcijo attr
\begin{Schunk}
\begin{Sinput}
 attr(X, "scaled:center")
\end{Sinput}
\begin{Soutput}
        x         y 
0.6666667 3.0000000 
\end{Soutput}
\end{Schunk}
\end{frame}
%% <<<<<<---------------------|

\begin{Schunk}
\begin{Sinput}
 t(X)
\end{Sinput}
\begin{Soutput}
        [,1]      [,2]      [,3]
x  0.3333333  1.333333 -1.666667
y -1.0000000 -2.000000  3.000000
attr(,"scaled:center")
        x         y 
0.6666667 3.0000000 
\end{Soutput}
\begin{Sinput}
 X
\end{Sinput}
\begin{Soutput}
              x  y
[1,]  0.3333333 -1
[2,]  1.3333333 -2
[3,] -1.6666667  3
attr(,"scaled:center")
        x         y 
0.6666667 3.0000000 
\end{Soutput}
\end{Schunk}



%% |--------------------->>>>>>
\begin{frame}[fragile]
\frametitle{Matrika SSP}
\begin{Schunk}
\begin{Sinput}
 C <- t(X) %*% X
 C
\end{Sinput}
\begin{Soutput}
          x  y
x  4.666667 -8
y -8.000000 14
\end{Soutput}
\end{Schunk}
\end{frame}
%% <<<<<<---------------------|



%% |--------------------->>>>>>
\begin{frame}[fragile]
\frametitle{Kovariančna matrika}
\begin{Schunk}
\begin{Sinput}
 S <- C/(n - 1)
 S
\end{Sinput}
\begin{Soutput}
          x  y
x  2.333333 -4
y -4.000000  7
\end{Soutput}
\begin{Sinput}
 cov(X)
\end{Sinput}
\begin{Soutput}
          x  y
x  2.333333 -4
y -4.000000  7
\end{Soutput}
\begin{Sinput}
 diag(S)
\end{Sinput}
\begin{Soutput}
       x        y 
2.333333 7.000000 
\end{Soutput}
\end{Schunk}
\end{frame}
%% <<<<<<---------------------|



%% |--------------------->>>>>>
\begin{frame}[fragile]
\frametitle{Matrika varianc}
\begin{Schunk}
\begin{Sinput}
 diag(diag(S))
\end{Sinput}
\begin{Soutput}
         [,1] [,2]
[1,] 2.333333    0
[2,] 0.000000    7
\end{Soutput}
\begin{Sinput}
 SD1 <- sqrt(diag(1/diag(S)))
 SD1
\end{Sinput}
\begin{Soutput}
          [,1]      [,2]
[1,] 0.6546537 0.0000000
[2,] 0.0000000 0.3779645
\end{Soutput}
\end{Schunk}
\end{frame}
%% <<<<<<---------------------|



%% |--------------------->>>>>>
\begin{frame}[fragile]
\frametitle{Korelacijska matrika}
\begin{Schunk}
\begin{Sinput}
 R <- SD1 %*% S %*% SD1
 R
\end{Sinput}
\begin{Soutput}
           [,1]       [,2]
[1,]  1.0000000 -0.9897433
[2,] -0.9897433  1.0000000
\end{Soutput}
\end{Schunk}
\end{frame}
%% <<<<<<---------------------|



%% |--------------------->>>>>>
\begin{frame}[fragile]
\frametitle{Še enkrat}
\begin{Schunk}
\begin{Sinput}
 SX <- scale(X)
 t(SX) %*% SX/(n - 1)
\end{Sinput}
\begin{Soutput}
           x          y
x  1.0000000 -0.9897433
y -0.9897433  1.0000000
\end{Soutput}
\begin{Sinput}
 cor(X)
\end{Sinput}
\begin{Soutput}
           x          y
x  1.0000000 -0.9897433
y -0.9897433  1.0000000
\end{Soutput}
\end{Schunk}
\end{frame}
%% <<<<<<---------------------|



Opisne statistike za matriko lahko dobimo na različne načine.
Centroid (vektor povprečij) lahko izvlečemo s pomočjo funkcije
\code{colMeans}, s funkcijo \code{scale}, lahko pa tudi z
\code{apply}:

%% |--------------------->>>>>>
\begin{frame}[fragile]
\frametitle{Centroidi na več načinov}
\begin{Schunk}
\begin{Sinput}
 (X <- cbind(x, y))
\end{Sinput}
\begin{Soutput}
      x y
[1,]  1 2
[2,]  2 1
[3,] -1 6
\end{Soutput}
\begin{Sinput}
 (colMeans(X))
\end{Sinput}
\begin{Soutput}
        x         y 
0.6666667 3.0000000 
\end{Soutput}
\begin{Sinput}
 (attr(scale(X), "scaled:center"))
\end{Sinput}
\begin{Soutput}
        x         y 
0.6666667 3.0000000 
\end{Soutput}
\begin{Sinput}
 (apply(X, 2, mean))
\end{Sinput}
\begin{Soutput}
        x         y 
0.6666667 3.0000000 
\end{Soutput}
\end{Schunk}
\end{frame}
%% <<<<<<---------------------|

Za dodatne funkcije si oglejte \code{help(colMeans)}.

%% |--------------------->>>>>>
\begin{frame}[fragile]
\frametitle{Standardni odkloni}
Standardne odklone bi lahko izvlekli kot koren diagonal
kovariančne matrike, kot atribut \code{scale} ali pa z \code{apply}:

\begin{Schunk}
\begin{Sinput}
 (sqrt(diag(var(X))))
\end{Sinput}
\begin{Soutput}
       x        y 
1.527525 2.645751 
\end{Soutput}
\begin{Sinput}
 (attr(scale(X), "scaled:scale"))
\end{Sinput}
\begin{Soutput}
       x        y 
1.527525 2.645751 
\end{Soutput}
\begin{Sinput}
 (apply(X, 2, sd))
\end{Sinput}
\begin{Soutput}
       x        y 
1.527525 2.645751 
\end{Soutput}
\end{Schunk}
\end{frame}
%% <<<<<<---------------------|


\subsection{Funkcije za linearno algebro}

%% |--------------------->>>>>>
\begin{frame}[fragile]
\frametitle{Funkcije za linearno algebro}
Na voljo imamo funkcije, za manipulacijo matrik in pomembne operacije nad matrikami
\begin{itemize*}
  \item \fct{t} transponiranje matrik
  \item \fct{eigen} lastne vrednosti
  \item \fct{solve} inverzna matrika
  \item \fct{det} determinanta matrike
  \item sled izračunamo kot vsoto diagonalnih elementov
\end{itemize*}
\end{frame}
%% <<<<<<---------------------|


%% |--------------------->>>>>>
\begin{frame}[fragile]
\frametitle{Lastni vektorji}
\begin{Schunk}
\begin{Sinput}
 S <- matrix(c(10, 3, 3, 2), 2, 2)
 S
\end{Sinput}
\begin{Soutput}
     [,1] [,2]
[1,]   10    3
[2,]    3    2
\end{Soutput}
\begin{Sinput}
 eigen(S)
\end{Sinput}
\begin{Soutput}
$values
[1] 11  1

$vectors
           [,1]       [,2]
[1,] -0.9486833  0.3162278
[2,] -0.3162278 -0.9486833
\end{Soutput}
\begin{Sinput}
 det(S)
\end{Sinput}
\begin{Soutput}
[1] 11
\end{Soutput}
\end{Schunk}
\end{frame}
%% <<<<<<---------------------|


%% |--------------------->>>>>>
\begin{frame}[fragile]
\frametitle{Inverzna matrika}
\begin{Schunk}
\begin{Sinput}
 S1 <- solve(S)
 S %*% S1
\end{Sinput}
\begin{Soutput}
              [,1] [,2]
[1,]  1.000000e+00    0
[2,] -2.220446e-16    1
\end{Soutput}
\end{Schunk}
Takole se znebimo zelo majhnih vrednosti v izpisu

\begin{Schunk}
\begin{Sinput}
 zapsmall(S %*% S1)
\end{Sinput}
\begin{Soutput}
     [,1] [,2]
[1,]    1    0
[2,]    0    1
\end{Soutput}
\end{Schunk}

Sled bi lahko izračunali kot

\begin{Schunk}
\begin{Sinput}
 sum(diag(S))
\end{Sinput}
\begin{Soutput}
[1] 12
\end{Soutput}
\end{Schunk}

\end{frame}
%% <<<<<<---------------------|

Primerjajte determinanto in sled z lastnima vrednostma.

\begin{Schunk}
\begin{Sinput}
 names(data)
\end{Sinput}
\begin{Soutput}
 [1] "starost" "mesec"   "spol"    "masa"    "visina"  "roke"   
 [7] "cevelj"  "lasje"   "oci"     "mati"    "oce"     "majica" 
\end{Soutput}
\end{Schunk}

%% |--------------------->>>>>>
\begin{frame}[fragile]
\frametitle{}
\begin{itemize*}
  \item Iz podatkov \code{data} izberite nekaj (3 ali 4) številske spremenljivke in formirajte matriko \code{X}
    \item Narišite pare razsevnih diagramov - funkcija \fct{pairs}
  \item Poiščite povprečne vrednosti spremenljivk
  \item Izračunajte kovariančno in korelacijsko matriko
\end{itemize*}
\end{frame}
%% <<<<<<---------------------|

%% |--------------------->>>>>>
\begin{frame}[fragile]
\frametitle{Podatki}
\begin{Schunk}
\begin{Sinput}
 X <- data[, c("masa", "visina", "mesec", "roke", 
+     "cevelj")]
 head(X)
\end{Sinput}
\begin{Soutput}
  masa visina mesec roke cevelj
2   60    173     1  176     43
3   55    178     7  178     39
4   70    167     8  165     39
5   65    171     4  168     40
6   88    171     3  173     41
7   52    162     7  164     39
\end{Soutput}
\begin{Sinput}
 tail(X)
\end{Sinput}
\begin{Soutput}
   masa visina mesec roke cevelj
38   73    173    10  180     42
39   58    170    10  171     48
40   56    158     7  156     37
41   55    157     4   NA     37
42   50    160    10  160     37
43   73    181    NA  187     43
\end{Soutput}
\end{Schunk}
\end{frame}
%% <<<<<<---------------------|

%% |--------------------->>>>>>
\begin{frame}[fragile]
\frametitle{}
\begin{Schunk}
\begin{Sinput}
 cor(X, use = "complete.obs")
\end{Sinput}
\begin{Soutput}
             masa    visina      mesec      roke    cevelj
masa   1.00000000 0.6786316 0.03182096 0.6657479 0.5474904
visina 0.67863160 1.0000000 0.17764477 0.9393966 0.6904270
mesec  0.03182096 0.1776448 1.00000000 0.1609260 0.1037727
roke   0.66574793 0.9393966 0.16092600 1.0000000 0.7225814
cevelj 0.54749044 0.6904270 0.10377266 0.7225814 1.0000000
\end{Soutput}
\end{Schunk}
\end{frame}
%% <<<<<<---------------------|


\section{Test hipotez}

%% |--------------------->>>>>>
\begin{frame}[fragile]
\frametitle{Test hipotez}
Ali so fantje večji od deklet
\begin{Schunk}
\begin{Sinput}
 t.test(visina ~ spol, data = data)
\end{Sinput}
\begin{Soutput}
	Welch Two Sample t-test

data:  visina by spol 
t = -6.4643, df = 12.502, p-value = 2.55e-05
alternative hypothesis: true difference in means is not equal to 0 
95 percent confidence interval:
 -17.901862  -8.906219 
sample estimates:
mean in group F mean in group M 
       166.8182        180.2222 
\end{Soutput}
\end{Schunk}
\end{frame}
%% <<<<<<---------------------|


%% |--------------------->>>>>>
\begin{frame}[fragile]
\frametitle{Funkcija za Student t-test}
Ali so fantje večji od deklet
\begin{Schunk}
\begin{Sinput}
 student <- function(x, y) {
+ }
\end{Sinput}
\end{Schunk}
\end{frame}
%% <<<<<<---------------------|


%% |--------------------->>>>>>
\begin{frame}[fragile]
\frametitle{Permutacijski test}
$$H_0: \mu_1=\mu_2(=\mu)$$
Če imamo opravka z vzorcema iz iste populacije, potem je lahgko v vsakem vzorcu katerakoli od izmerjenih $n_1+n_2$ vrednosti. Če imam npr. 10
\begin{Schunk}
\begin{Sinput}
 n
\end{Sinput}
\begin{Soutput}
[1] 10
\end{Soutput}
\begin{Sinput}
 sample(10, 4)
\end{Sinput}
\begin{Soutput}
[1]  9  8 10  2
\end{Soutput}
\end{Schunk}
lahko takole določim, katere vrednosti spadajo v prvi vzorec.
\end{frame}
%% <<<<<<---------------------|

%% |--------------------->>>>>>
\begin{frame}[fragile]
\frametitle{Podatki}
Zgenerirajmo podatke iz populacije s povprečjem 10
\begin{Schunk}
\begin{Sinput}
 set.seed(555)
 n1 <- 5
 mu1 <- 10
 sd1 <- 1
 n2 <- 5
 mu2 <- 10
 sd2 <- 1
 n <- n1 + n2
 X1 <- round(rnorm(n1, mu1, sd1), 1)
 X2 <- round(rnorm(n2, mu2, sd2), 1)
 X1
\end{Sinput}
\begin{Soutput}
[1]  9.7 10.5 10.4 11.9  8.2
\end{Soutput}
\begin{Sinput}
 X2
\end{Sinput}
\begin{Soutput}
[1] 10.9  9.8 11.4 10.0 10.6
\end{Soutput}
\begin{Sinput}
 X <- c(X1, X2)
 ind <- c(rep(1, n1), rep(2, n2))
\end{Sinput}
\end{Schunk}
\end{frame}
%% <<<<<<---------------------|

%% |--------------------->>>>>>
\begin{frame}[fragile]
\frametitle{Funkcija za razliko povprečij}

\begin{Schunk}
\begin{Sinput}
 dx <- function(x, y) mean(y) - mean(x)
 dx(X1, X2)
\end{Sinput}
\begin{Soutput}
[1] 0.4
\end{Soutput}
\end{Schunk}
\end{frame}
%% <<<<<<---------------------|

%% |--------------------->>>>>>
\begin{frame}[fragile]
\frametitle{Premešajmo podatke}
\begin{Schunk}
\begin{Sinput}
 set.seed(432)
 (smp1 <- sample(n, n1))
\end{Sinput}
\begin{Soutput}
[1] 3 1 8 2 5
\end{Soutput}
\begin{Sinput}
 (smp2 <- (1:n)[-smp1])
\end{Sinput}
\begin{Soutput}
[1]  4  6  7  9 10
\end{Soutput}
\end{Schunk}
\end{frame}
%% <<<<<<---------------------|

%% |--------------------->>>>>>
\begin{frame}[fragile]
\frametitle{Premestitev vrednosti}
\begin{Schunk}
\begin{Soutput}
[1] 0.6
\end{Soutput}
\end{Schunk}
\includegraphics{./figs/SwPres-vzorca}
\end{frame}
%% <<<<<<---------------------|

%% |--------------------->>>>>>
\begin{frame}[fragile]
\frametitle{Kako je narisano}
\begin{Schunk}
\begin{Sinput}
 plot(X, rep(0, n), col = ind, pch = 16, ylab = "")
 points(X[smp1], rep(-1, n1))
 points(X[smp2], rep(1, n2))
 arrows(X[smp1], rep(0, n1), X[smp1], rep(-1, n1), 
+     col = ind[smp1])
 arrows(X[smp2], rep(0, n2), X[smp2], rep(1, n2), 
+     col = ind[smp2])
 abline(v = mean(X[smp1]), col = 4)
 abline(v = mean(X[smp2]), col = 4)
 points(c(mean(X[smp1]), mean(X[smp2])), c(-1, 1), 
+     pch = 16, col = 4)
 points(c(mean(X1), mean(X2)), c(0, 0), col = 4, cex = 2)
 dx(X[smp1], X[smp2])
\end{Sinput}
\begin{Soutput}
[1] 0.6
\end{Soutput}
\end{Schunk}
\end{frame}
%% <<<<<<---------------------|

%% |--------------------->>>>>>
\begin{frame}[fragile]
\frametitle{Funkcija}
\begin{Schunk}
\begin{Sinput}
 perm.test <- function(x, X1, X2) {
+     n1 <- length(X1)
+     n2 <- length(X2)
+     ind <- c(rep(1, n1), rep(2, n2))
+     (smp1 <- sample(n, n1))
+     (smp2 <- (1:n)[-smp1])
+     plot(X, rep(0, n), col = ind, pch = 16, ylab = "")
+     points(X[smp1], rep(-1, n1))
+     points(X[smp2], rep(1, n2))
+     arrows(X[smp1], rep(0, n1), X[smp1], rep(-1, 
+         n1), col = ind[smp1])
+     arrows(X[smp2], rep(0, n2), X[smp2], rep(1, n2), 
+         col = ind[smp2])
+     abline(v = mean(X[smp1]), col = 4)
+     abline(v = mean(X[smp2]), col = 4)
+     points(c(mean(X[smp1]), mean(X[smp2])), c(-1, 
+         1), pch = 16, col = 4)
+     points(c(mean(X1), mean(X2)), c(0, 0), col = 4, 
+         cex = 2)
+     return(dx(X[smp1], X[smp2]))
+ }
\end{Sinput}
\end{Schunk}
\end{frame}
%% <<<<<<---------------------|

%% |--------------------->>>>>>
\begin{frame}[fragile]
\frametitle{}

\begin{Schunk}
\begin{Sinput}
 perm.test(x, X1, X2)
\end{Sinput}
\begin{Soutput}
[1] 0.68
\end{Soutput}
\begin{Sinput}
 sapply(1:8, FUN = perm.test, X1 = X1, X2 = X2)
\end{Sinput}
\begin{Soutput}
[1]  0.88  0.08  0.04  1.24  0.16  0.12  0.16 -0.04
\end{Soutput}
\begin{Sinput}
 dx(X1, X2)
\end{Sinput}
\begin{Soutput}
[1] 0.4
\end{Soutput}
\end{Schunk}

\end{frame}
%% <<<<<<---------------------|









% ----------------------------------------------------------------
\bibliographystyle{amsplain}
\bibliography{ab-general}
\end{document}
% ----------------------------------------------------------------
