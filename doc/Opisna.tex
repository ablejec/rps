% -*- TeX:Rnw -*-
% ----------------------------------------------------------------
% .R Sweave file  ************************************************
% ----------------------------------------------------------------
%%
% \VignetteIndexEntry{}
% \VignetteDepends{}
% \VignettePackage{}
%\documentclass[a4paper,12pt]{article}
\usepackage[slovene]{babel}
\usepackage[utf8]{inputenc} %% must be here for Sweave encoding check
\newcommand{\SVNRevision}{$ $Rev: 3 $ $}
%\newcommand{\SVNDate}{$ $Date:: 2009-02-2#$ $}
\newcommand{\SVNId}{$ $Id: program.Rnw 3 2009-02-22 17:36:08Z ABlejec $ $}
%\usepackage{babel}
%\input{abpkgB}
%\input{abpkg}
\input{abBeam}
\input{abcmd}
%\input{abpage}
\usepackage{pgf,pgfarrows,pgfnodes,pgfautomata,pgfheaps,pgfshade}
\usepackage{amsmath,amssymb}
\usepackage{colortbl}
\usepackage{Sweave}
\input{mysweaveB}
\newcommand{\BV}{}
\newcommand{\EV}{}
\newcommand{\myemph}[1]{{\color{Sgreen} \textit{#1}}}
\input{./figs/SwPres-concordance}
%\SweaveOpts{echo=false}
%\usepackage{lmodern}
%\input{abfont}
%\SweaveOpts{keep.source=true}
% ----------------------------------------------------------------
\title{RPS\\Analiza podatkov}
\author{A. Blejec}
%\address{}%
%\email{}%
%
%\thanks{}%
%\subjclass{}%
%\keywords{}%

%\date{}%
%\dedicatory{}%
%\commby{}%
\begin{document}
\mode<article> {\maketitle}
\mode<presentation> {\frame{\titlepage}}
\tableofcontents
% ----------------------------------------------------------------
\begin{abstract}
 Primer analize podatkov
\end{abstract}
% -------------------------------------------------------------
%% Sweave settings for includegraphics default plot size (Sweave default is 0.8)
%% notice this must be after begin{document}
% \setkeys{Gin}{width=0.9\textwidth}
\setkeys{Gin}{width=0.7\textwidth}
% ----------------------------------------------------------------
O rasti in velikosti ljudi imamo nekaj mnenj, ki jih lahko izrazimo v obliki raziskovalnih vprašanj. Najprej si zastavimo vprašanja.
%% |--------------------->>>>>>
\begin{frame}[fragile]
\frametitle{Vprašanja}
Nekaj vprašanj, na katere bi radi odgovorili je:
\begin{itemize}
  \item Ali so fantje večji od deklet?
  \item Ali so fantje težji od deklet?
  \item Ali sta razpon rok in višina približno enaka?
  \item Ali drži Galtonovo opažanje glede višine otrok in staršev?
  \item ...
\end{itemize}
Zbrali smo nekaj podatkov o študentih, s katerimi si bomo lahko poskusili odgovoriti.
\end{frame}
%% <<<<<<---------------------|
Nato zberemo podatke, s katerimi bomo poskusili odgovoriti na vprašanja. Ker predvidevamo, da nas bo zanimalo še kaj, zberemo podatke o še nekaj spremenljivkah.
\begin{Schunk}
\begin{Sinput}
> lfn <- "Podatki2012.txt"
> 
\end{Sinput}
\end{Schunk}

%% |--------------------->>>>>>
\begin{frame}[fragile]
\frametitle{Podatki}
Podaki so o študentih 3. letnika biologije v letu 2012/13 so v datoteki \file{lfn}.
\begin{Schunk}
\begin{Sinput}
> fpath <- file.path("../data",lfn)
> data <- read.table(fpath,header=TRUE,sep="\t")
> names(data)
\end{Sinput}
\begin{Soutput}
 [1] "starost" "mesec"   "spol"    "masa"    "visina"  "roke"   
 [7] "cevelj"  "lasje"   "oci"     "mati"    "oce"     "majica" 
\end{Soutput}
\end{Schunk}
\end{frame}
%% <<<<<<---------------------|








% ----------------------------------------------------------------
%\bibliographystyle{amsplain}
%\bibliography{ab-general}
\end{document}
% ----------------------------------------------------------------
