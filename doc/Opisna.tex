% -*- TeX:Rnw -*-
% ----------------------------------------------------------------
% .R Sweave file  ************************************************
% ----------------------------------------------------------------
%%
% \VignetteIndexEntry{}
% \VignetteDepends{}
% \VignettePackage{}
%\documentclass[a4paper,12pt]{article}
\usepackage[slovene]{babel}
\usepackage[utf8]{inputenc} %% must be here for Sweave encoding check
\newcommand{\SVNRevision}{$ $Rev: 3 $ $}
%\newcommand{\SVNDate}{$ $Date:: 2009-02-2#$ $}
\newcommand{\SVNId}{$ $Id: program.Rnw 3 2009-02-22 17:36:08Z ABlejec $ $}
%\usepackage{babel}
%\input{abpkgB}
%\input{abpkg}
\input{abBeam}
\input{abcmd}
%\input{abpage}
\usepackage{pgf,pgfarrows,pgfnodes,pgfautomata,pgfheaps,pgfshade}
\usepackage{amsmath,amssymb}
\usepackage{colortbl}
\usepackage{Sweave}
\input{mysweaveB}
\newcommand{\BV}{}
\newcommand{\EV}{}
\newcommand{\myemph}[1]{{\color{Sgreen} \textit{#1}}}
\input{./figs/RPS-concordance}
%\SweaveOpts{echo=false}
%\usepackage{lmodern}
%\input{abfont}

% ----------------------------------------------------------------
\title{RPS\\Analiza podatkov}
\author{A. Blejec}
%\address{}%
%\email{}%
%
%\thanks{}%
%\subjclass{}%
%\keywords{}%

%\date{}%
%\dedicatory{}%
%\commby{}%
\begin{document}
\mode<article> {\maketitle}
\mode<presentation> {\frame{\titlepage}}
\tableofcontents
% ----------------------------------------------------------------
\begin{abstract}
 Primer analize podatkov
\end{abstract}
% -------------------------------------------------------------
%% Sweave settings for includegraphics default plot size (Sweave default is 0.8)
%% notice this must be after begin{document}
% \setkeys{Gin}{width=0.9\textwidth}
\setkeys{Gin}{width=0.7\textwidth}
% ----------------------------------------------------------------
O rasti in velikosti ljudi imamo nekaj mnenj, ki jih lahko izrazimo v obliki raziskovalnih vprašanj. Najprej si zastavimo vprašanja.
%% |--------------------->>>>>>
\begin{frame}[fragile]
\frametitle{Vprašanja}
Nekaj vprašanj, na katere bi radi odgovorili je:
\begin{itemize}
  \item Ali so fantje večji od deklet?
  \item Ali so fantje težji od deklet?
  \item Ali sta razpon rok in višina približno enaka?
  \item Ali drži Galtonovo opažanje glede višine otrok in staršev?
  \item ...
\end{itemize}
Zbrali smo nekaj podatkov o študentih, s katerimi si bomo lahko poskusili odgovoriti.
\end{frame}
%% <<<<<<---------------------|
Nato zberemo podatke, s katerimi bomo poskusili odgovoriti na vprašanja. Ker predvidevamo, da nas bo zanimalo še kaj, zberemo podatke o še nekaj spremenljivkah.
\begin{Schunk}
\begin{Sinput}
  lfn <- "Podatki2012.txt"
\end{Sinput}
\end{Schunk}

%% |--------------------->>>>>>
\begin{frame}[fragile]
\frametitle{Podatki}
Podaki so o študentih 3. letnika biologije v letu 2012/13 so v datoteki \file{lfn} in na \url{http://bit.ly/16oBVpR}
\begin{Schunk}
\begin{Sinput}
  fpath <- "http://bit.ly/16oBVpR"
  data <- read.table(fpath, header = TRUE, sep = "\t")
  names(data)
\end{Sinput}
\begin{Soutput}
 [1] "starost" "mesec"   "spol"    "masa"    "visina" 
 [6] "roke"    "cevelj"  "lasje"   "oci"     "mati"   
[11] "oce"     "majica" 
\end{Soutput}
\end{Schunk}
\end{frame}
%% <<<<<<---------------------|

%% |--------------------->>>>>>
\begin{frame}[fragile]
\frametitle{Opisna statistika}
\begin{Schunk}
\begin{Sinput}
  summary(data[, 1:6])
\end{Sinput}
\begin{Soutput}
    starost          mesec        spol        masa      
 Min.   :20.00   Min.   : 0.000   F:33   Min.   :50.00  
 1st Qu.:21.00   1st Qu.: 5.000   M:10   1st Qu.:55.50  
 Median :21.00   Median : 7.000          Median :61.00  
 Mean   :22.07   Mean   : 6.814          Mean   :63.42  
 3rd Qu.:22.00   3rd Qu.: 9.500          3rd Qu.:70.00  
 Max.   :59.00   Max.   :11.000          Max.   :91.00  
                                                        
     visina           roke      
 Min.   :156.0   Min.   :154.0  
 1st Qu.:164.0   1st Qu.:163.2  
 Median :170.0   Median :167.8  
 Mean   :169.9   Mean   :169.3  
 3rd Qu.:173.5   3rd Qu.:172.5  
 Max.   :189.0   Max.   :193.0  
                 NA's   :5      
\end{Soutput}
\end{Schunk}
Ali pri podatkih kaj opazite?
\end{frame}
%% <<<<<<---------------------|
%% |--------------------->>>>>>
\begin{frame}[fragile]
\frametitle{Nenavadni podatki}
Kaj storiti s tistim, ki je napisal, da je rojen v mesecu 0?

Eden pa je star 59 let??
\end{frame}
%% <<<<<<---------------------|

%% |--------------------->>>>>>
\begin{frame}[fragile]
\frametitle{Popravljanje podatkov}
Odstranimo ta starga
\begin{Schunk}
\begin{Sinput}
  data <- data[data$starost < 30, ]
\end{Sinput}
\end{Schunk}
Podatke o mesecu 0 spremenimo v \code{NA}

\begin{Schunk}
\begin{Sinput}
  data[data$mesec == 0, "mesec"] <- NA
  table(data$mesec)
\end{Sinput}
\begin{Soutput}
 1  2  3  4  5  6  7  8  9 10 11 
 1  3  2  3  4  3  7  5  2  5  6 
\end{Soutput}
\end{Schunk}

\end{frame}
%% <<<<<<---------------------|



%% |--------------------->>>>>>
\begin{frame}[fragile]
\frametitle{Nadaljevanje opisa}
\begin{Schunk}
\begin{Sinput}
  summary(data[, 7:dim(data)[2]])
\end{Sinput}
\begin{Soutput}
     cevelj      lasje  oci         mati      
 Min.   :36.00   S:19   S:23   Min.   :157.0  
 1st Qu.:38.00   T:23   T:19   1st Qu.:160.0  
 Median :39.00                 Median :165.0  
 Mean   :39.93                 Mean   :165.6  
 3rd Qu.:41.00                 3rd Qu.:168.0  
 Max.   :48.00                 Max.   :180.0  
                               NA's   :5      
      oce        majica 
 Min.   :170.0   L : 4  
 1st Qu.:174.0   M :19  
 Median :179.0   S :16  
 Mean   :179.1   XL: 1  
 3rd Qu.:182.0   XS: 2  
 Max.   :190.0          
 NA's   :5              
\end{Soutput}
\end{Schunk}
\end{frame}
%% <<<<<<---------------------|
\clearpage
\section{Višina in spol}

Primerjajte razpone vrednosti višin študentov in staršev.

%% |--------------------->>>>>>
\begin{frame}[fragile]
\frametitle{Višina po spolu}
Povzetek višin glede na spol
\begin{Schunk}
\begin{Sinput}
  summary(data$mati)
\end{Sinput}
\begin{Soutput}
   Min. 1st Qu.  Median    Mean 3rd Qu.    Max.    NA's 
  157.0   160.0   165.0   165.6   168.0   180.0       5 
\end{Soutput}
\begin{Sinput}
  by(data$visina, data$spol, summary)
\end{Sinput}
\begin{Soutput}
data$spol: F
   Min. 1st Qu.  Median    Mean 3rd Qu.    Max. 
  156.0   163.0   168.0   166.8   170.0   178.0 
--------------------------------------------- 
data$spol: M
   Min. 1st Qu.  Median    Mean 3rd Qu.    Max. 
  171.0   180.0   180.0   180.2   183.0   189.0 
\end{Soutput}
\begin{Sinput}
  summary(data$oce)
\end{Sinput}
\begin{Soutput}
   Min. 1st Qu.  Median    Mean 3rd Qu.    Max.    NA's 
  170.0   174.0   179.0   179.1   182.0   190.0       5 
\end{Soutput}
\end{Schunk}
\end{frame}
%% <<<<<<---------------------|
%% |--------------------->>>>>>
\begin{frame}[fragile]
\frametitle{Doseg  spremenljivk v objektu \code{data.frame}}
Poglejte kakšne so vrednosti spremenljivke \code{visina}!\\
Ali je v delovnem prostoru (workspace)?\\
Do spremenljivk lahko pridem posredno na več načinov
\begin{itemize}
  \item \code{data\$visina}
  \item \code{data[, 'visina']}
  \item \code{data[,5]}
\end{itemize}
\end{frame}
%% <<<<<<---------------------|
%% |--------------------->>>>>>
\begin{frame}[fragile]
\frametitle{Neposreden dostop}
Neposreden dostop do spremenljivk omogoči
\begin{Schunk}
\begin{Sinput}
  attach(data)
  length(visina)
\end{Sinput}
\begin{Soutput}
[1] 42
\end{Soutput}
\begin{Sinput}
  visina[1:5]
\end{Sinput}
\begin{Soutput}
[1] 173 178 167 171 171
\end{Soutput}
\end{Schunk}
\end{frame}
%% <<<<<<---------------------|


Grafični prikazi
%% |--------------------->>>>>>
\begin{frame}[fragile]
\frametitle{Grafični prikaz podatkov}
\mode<presentation> {\mode<presentation> {\vspace{-1cm}}}
\begin{Schunk}
\begin{Sinput}
  plot(visina)
\end{Sinput}
\end{Schunk}
\includegraphics{./figs/RPS-011}
\end{frame}
%% <<<<<<---------------------|
%% |--------------------->>>>>>
\begin{frame}[fragile]
\frametitle{Grafični prikaz podatkov}
\mode<presentation> {\vspace{-1cm}}
\begin{Schunk}
\begin{Sinput}
  plot(visina, pch = 16, col = spol)
\end{Sinput}
\end{Schunk}
\includegraphics{./figs/RPS-012}
\end{frame}
%% <<<<<<---------------------|
%% |--------------------->>>>>>
\begin{frame}[fragile]
\frametitle{Kumulativa}
\mode<presentation> {\vspace{-1cm}}
\begin{Schunk}
\begin{Sinput}
  x <- visina
  plot(x, rank(x), pch = 16, col = spol)
\end{Sinput}
\end{Schunk}
\includegraphics{./figs/RPS-013}
\end{frame}
%% <<<<<<---------------------|

%% |--------------------->>>>>>
\begin{frame}[fragile]
\frametitle{Kumulativa in normalna aproksimacija}
\mode<presentation> {\vspace{-1cm}}
\begin{Schunk}
\begin{Sinput}
  x <- visina
  plot(x, rank(x), pch = 16, col = spol)
  q <- qnorm((rank(x) - 0.5)/length(x), mean(x), 
+     sd(x))
  points(q, rank(x), col = 4)
  segments(x, rank(x), q, rank(x))
\end{Sinput}
\end{Schunk}
\includegraphics{./figs/RPS-014}
\end{frame}
%% <<<<<<---------------------|
%% |--------------------->>>>>>
\begin{frame}[fragile]
\frametitle{Slika kvantilov}
\mode<presentation> {\vspace{-1cm}}
\begin{Schunk}
\begin{Sinput}
  qqnorm(visina, col = spol, pch = 16)
  qqline(visina)
\end{Sinput}
\end{Schunk}
\includegraphics{./figs/RPS-015}
\end{frame}
%% <<<<<<---------------------|
%% |--------------------->>>>>>
\begin{frame}[fragile]
\frametitle{Boxplot}
\mode<presentation> {\vspace{-1cm}}
\begin{Schunk}
\begin{Sinput}
  boxplot(visina ~ spol, col = c("pink", "lightblue"))
  rug(jitter(visina[spol == "F"]), side = 2)
  rug(jitter(visina[spol == "M"]), side = 4)
\end{Sinput}
\end{Schunk}
\includegraphics{./figs/RPS-016}

Dorišite točke za mediane. Pomagajte si s \fct{str}, \fct{locator}.
\end{frame}
%% <<<<<<---------------------|

\section{Testiranje višin}

%% |--------------------->>>>>>
\begin{frame}[fragile]
\frametitle{Student t-test}
\begin{Schunk}
\begin{Sinput}
  t.test(visina ~ spol)
\end{Sinput}
\begin{Soutput}
	Welch Two Sample t-test

data:  visina by spol 
t = -6.4643, df = 12.502, p-value = 2.55e-05
alternative hypothesis: true difference in means is not equal to 0 
95 percent confidence interval:
 -17.901862  -8.906219 
sample estimates:
mean in group F mean in group M 
       166.8182        180.2222 
\end{Soutput}
\end{Schunk}
Lahko tudi tako:
\begin{Schunk}
\begin{Sinput}
  t.test(visina[spol == "F"], visina[spol == "M"])
\end{Sinput}
\end{Schunk}
Oglejte si, kaj vrne funkcija \fct{t.test}. Dorišite točki povprečij.
\end{frame}

%% <<<<<<---------------------|
\clearpage
\section{Teža}
%% |--------------------->>>>>>
\begin{frame}[fragile]
\frametitle{Teža in spol}
Izberite si nekaj prejšnjih prikazov in
\begin{itemize}
  \item Raziščite kako je s težo pri dekletih in fantih.
  \item Izračunajte novo spremenljivko $BMI=masa/visina^2$
  \item Kaj pa velja za BMI?
\end{itemize}

\end{frame}
%% <<<<<<---------------------|

\begin{Schunk}
\begin{Sinput}
  t.test(masa ~ spol)
\end{Sinput}
\begin{Soutput}
	Welch Two Sample t-test

data:  masa by spol 
t = -7.0271, df = 10.154, p-value = 3.324e-05
alternative hypothesis: true difference in means is not equal to 0 
95 percent confidence interval:
 -25.71686 -13.35385 
sample estimates:
mean in group F mean in group M 
       58.57576        78.11111 
\end{Soutput}
\end{Schunk}

\begin{Schunk}
\begin{Sinput}
  boxplot(masa ~ spol, col = c("pink", "lightblue"))
  rug(jitter(masa[spol == "F"]), side = 2)
  rug(jitter(masa[spol == "M"]), side = 4)
\end{Sinput}
\end{Schunk}
\includegraphics{./figs/RPS-020}


\clearpage
\section{Galton in višina otrok in staršev}
%% |--------------------->>>>>>
\begin{frame}[fragile]
\frametitle{Velikost staršev in potomcev}
Galton je ugotavljal korelacijo med velikostjo staršev in potomcev.

Uvedel je pojem regresija, ki izvira iz ugotovitve, da so velikost staršev in potomcev v posebnem razmerju, ki zagotavlja 'regesijo' k povprečju.
\end{frame}
%% <<<<<<---------------------|

%% |--------------------->>>>>>
\begin{frame}[fragile]
\frametitle{Fantje}
\begin{Schunk}
\begin{Sinput}
  with(data, plot(oce, visina, col = spol, pch = 16, 
+     xlim = range(visina)))
  abline(c(0, 1), col = "blue")
  abline(lm(visina ~ oce, data = data), col = 3, 
+     lwd = 3)
  abline(lm(visina ~ oce, data = data[data$spol == 
+     "M", ]), col = "red", lwd = 3)
  abline(lm(visina ~ oce, data = data[data$spol == 
+     "F", ]), lwd = 3)
\end{Sinput}
\end{Schunk}
\end{frame}
%% <<<<<<---------------------|

%% |--------------------->>>>>>
\begin{frame}[fragile]
\frametitle{Koeficienti}
\begin{Schunk}
\begin{Sinput}
  fit <- lm(visina ~ oce, data = data)
  summary(fit)
\end{Sinput}
\begin{Soutput}
Call:
lm(formula = visina ~ oce, data = data)

Residuals:
    Min      1Q  Median      3Q     Max 
-14.298  -4.298  -1.343   3.998  16.315 

Coefficients:
            Estimate Std. Error t value Pr(>|t|)  
(Intercept)  83.4128    39.1670   2.130   0.0403 *
oce           0.4774     0.2186   2.183   0.0358 *
---
Signif. codes:  0 '***' 0.001 '**' 0.01 '*' 0.05 '.' 0.1 ' ' 1 

Residual standard error: 7.231 on 35 degrees of freedom
  (5 observations deleted due to missingness)
Multiple R-squared: 0.1199,	Adjusted R-squared: 0.09474 
F-statistic: 4.767 on 1 and 35 DF,  p-value: 0.0358 
\end{Soutput}
\end{Schunk}
\end{frame}
%% <<<<<<---------------------|

%% |--------------------->>>>>>
\begin{frame}[fragile]
\frametitle{Grafična analiza}
\begin{Schunk}
\begin{Sinput}
  par(mfrow = c(2, 2))
  plot(fit, col = spol)
\end{Sinput}
\end{Schunk}
\includegraphics{./figs/RPS-023}
\end{frame}
%% <<<<<<---------------------|


\begin{Schunk}
\begin{Sinput}
  fit <- lm(visina ~ spol * oce, data = data)
  summary(fit)
\end{Sinput}
\begin{Soutput}
Call:
lm(formula = visina ~ spol * oce, data = data)

Residuals:
     Min       1Q   Median       3Q      Max 
-11.3717  -2.6359  -0.2208   3.7604  11.1943 

Coefficients:
            Estimate Std. Error t value Pr(>|t|)   
(Intercept) 115.8618    34.2922   3.379  0.00188 **
spolM       -10.4524    72.3167  -0.145  0.88596   
oce           0.2830     0.1920   1.474  0.14986   
spolM:oce     0.1264     0.4003   0.316  0.75420   
---
Signif. codes:  0 '***' 0.001 '**' 0.01 '*' 0.05 '.' 0.1 ' ' 1 

Residual standard error: 5.475 on 33 degrees of freedom
  (5 observations deleted due to missingness)
Multiple R-squared: 0.5244,	Adjusted R-squared: 0.4812 
F-statistic: 12.13 on 3 and 33 DF,  p-value: 1.645e-05 
\end{Soutput}
\end{Schunk}

\begin{Schunk}
\begin{Sinput}
  par(mfrow = c(2, 2))
  plot(fit, col = spol)
\end{Sinput}
\end{Schunk}
\includegraphics{./figs/RPS-025}


%% |--------------------->>>>>>
\begin{frame}[fragile]
\frametitle{Regresija}
\begin{Schunk}
\begin{Sinput}
  plot(visina, masa)
  abline(lm(masa ~ visina))
\end{Sinput}
\end{Schunk}
\includegraphics{./figs/RPS-026}
\end{frame}
%% <<<<<<---------------------|

%% |--------------------->>>>>>
\begin{frame}[fragile]
\frametitle{Regresija}
\begin{Schunk}
\begin{Sinput}
  cor(masa, visina)
\end{Sinput}
\begin{Soutput}
[1] 0.7049331
\end{Soutput}
\begin{Sinput}
  fit <- lm(masa ~ visina)
  summary(fit)
\end{Sinput}
\begin{Soutput}
Call:
lm(formula = masa ~ visina)

Residuals:
    Min      1Q  Median      3Q     Max 
-15.354  -4.140  -1.786   3.579  24.042 

Coefficients:
            Estimate Std. Error t value Pr(>|t|)    
(Intercept) -92.2700    24.6887  -3.737 0.000581 ***
visina        0.9136     0.1453   6.286 1.87e-07 ***
---
Signif. codes:  0 '***' 0.001 '**' 0.01 '*' 0.05 '.' 0.1 ' ' 1 

Residual standard error: 7.202 on 40 degrees of freedom
Multiple R-squared: 0.4969,	Adjusted R-squared: 0.4844 
F-statistic: 39.51 on 1 and 40 DF,  p-value: 1.874e-07 
\end{Soutput}
\end{Schunk}
\end{frame}
%% <<<<<<---------------------|

%% |--------------------->>>>>>
\begin{frame}[fragile]
\frametitle{Regresija}
\begin{Schunk}
\begin{Sinput}
  fit <- lm(masa ~ visina * spol)
  summary(fit)
\end{Sinput}
\begin{Soutput}
Call:
lm(formula = masa ~ visina * spol)

Residuals:
   Min     1Q Median     3Q    Max 
-8.381 -4.148 -1.588  3.022 11.878 

Coefficients:
             Estimate Std. Error t value Pr(>|t|)  
(Intercept)  -13.1070    31.2526  -0.419   0.6773  
visina         0.4297     0.1872   2.295   0.0274 *
spolM        100.1884    73.0398   1.372   0.1782  
visina:spolM  -0.4795     0.4113  -1.166   0.2509  
---
Signif. codes:  0 '***' 0.001 '**' 0.01 '*' 0.05 '.' 0.1 ' ' 1 

Residual standard error: 5.738 on 38 degrees of freedom
Multiple R-squared: 0.6966,	Adjusted R-squared: 0.6727 
F-statistic: 29.09 on 3 and 38 DF,  p-value: 6.079e-10 
\end{Soutput}
\end{Schunk}
\end{frame}
%% <<<<<<---------------------|

%% |--------------------->>>>>>
\begin{frame}[fragile]
\frametitle{}
\begin{Schunk}
\begin{Sinput}
  plot(visina, masa, col = spol)
  abline(lm(masa ~ visina))
  points(visina, predict(fit), pch = 16, col = spol)
\end{Sinput}
\end{Schunk}
\includegraphics{./figs/RPS-029}
\end{frame}
%% <<<<<<---------------------|

%% |--------------------->>>>>>
\begin{frame}[fragile]
\frametitle{Analiza variance}
\begin{Schunk}
\begin{Sinput}
  fvis <- cut(visina, breaks = c(155, 165, 175, 
+     200), labels = c("M", "S", "V"))
  table(fvis)
\end{Sinput}
\begin{Soutput}
fvis
 M  S  V 
12 22  8 
\end{Soutput}
\begin{Sinput}
  (m <- by(masa, fvis, mean))
\end{Sinput}
\begin{Soutput}
fvis: M
[1] 56.16667
--------------------------------------------- 
fvis: S
[1] 62.04545
--------------------------------------------- 
fvis: V
[1] 74.625
\end{Soutput}
\end{Schunk}
\end{frame}
%% <<<<<<---------------------|

%% |--------------------->>>>>>
\begin{frame}[fragile]
\frametitle{AOV}
\begin{Schunk}
\begin{Sinput}
  plot(masa, fvis)
  points(m, 1:3, pch = 16, cex = 2)
\end{Sinput}
\end{Schunk}
\includegraphics{./figs/RPS-031}
\end{frame}
%% <<<<<<---------------------|

%% |--------------------->>>>>>
\begin{frame}[fragile]
\frametitle{AOV}
\begin{Schunk}
\begin{Sinput}
  fit <- lm(masa ~ 0 + fvis)
  summary(fit)
\end{Sinput}
\begin{Soutput}
Call:
lm(formula = masa ~ 0 + fvis)

Residuals:
    Min      1Q  Median      3Q     Max 
-19.625  -4.510  -1.167   2.924  25.954 

Coefficients:
      Estimate Std. Error t value Pr(>|t|)    
fvisM   56.167      2.295   24.48   <2e-16 ***
fvisS   62.045      1.695   36.61   <2e-16 ***
fvisV   74.625      2.811   26.55   <2e-16 ***
---
Signif. codes:  0 '***' 0.001 '**' 0.01 '*' 0.05 '.' 0.1 ' ' 1 

Residual standard error: 7.949 on 39 degrees of freedom
Multiple R-squared: 0.9855,	Adjusted R-squared: 0.9843 
F-statistic: 881.4 on 3 and 39 DF,  p-value: < 2.2e-16 
\end{Soutput}
\end{Schunk}
\end{frame}
%% <<<<<<---------------------|






%% |--------------------->>>>>>
\begin{frame}[fragile]
\frametitle{Fantje}
\includegraphics{./figs/RPS-moski}
\end{frame}
%% <<<<<<---------------------|
\clearpage
%% |--------------------->>>>>>
\begin{frame}[fragile]
\frametitle{Dekleta}
\begin{Schunk}
\begin{Sinput}
  with(data, plot(mati, visina, col = spol, pch = 16, 
+     xlim = range(visina)))
  abline(c(0, 1), col = "blue")
  abline(lm(visina ~ mati, data = data), col = 3, 
+     lwd = 3)
  abline(lm(visina ~ mati, data = data[data$spol == 
+     "M", ]), col = "red", lwd = 3)
  abline(lm(visina ~ mati, data = data[data$spol == 
+     "F", ]), lwd = 3)
\end{Sinput}
\end{Schunk}
\end{frame}
%% <<<<<<---------------------|

%% |--------------------->>>>>>
\begin{frame}[fragile]
\frametitle{Dekleta}
\includegraphics{./figs/RPS-zenske}
\end{frame}
%% <<<<<<---------------------|
\clearpage
\setkeys{Gin}{width=0.9\textwidth}
%% |--------------------->>>>>>
\begin{frame}[fragile]
\frametitle{Fantje in dekleta}
\includegraphics{./figs/RPS-moski2}
\end{frame}
%% <<<<<<---------------------|
 \setkeys{Gin}{width=0.7\textwidth}



\clearpage
\section*{SessionInfo}
{\small
Windows 7 x64 (build 7601) Service Pack 1 \begin{itemize}\raggedright
  \item R version 2.15.1 (2012-06-22), \verb|x86_64-pc-mingw32|
  \item Locale: \verb|LC_COLLATE=Slovenian_Slovenia.1250|, \verb|LC_CTYPE=Slovenian_Slovenia.1250|, \verb|LC_MONETARY=Slovenian_Slovenia.1250|, \verb|LC_NUMERIC=C|, \verb|LC_TIME=Slovenian_Slovenia.1250|
  \item Base packages: base, datasets, graphics,
    grDevices, stats, utils
  \item Other packages: patchDVI~1.9
  \item Loaded via a namespace (and not attached):
    tools~2.15.1
\end{itemize}Project path:\verb' D:/_Y/R/rps '\\Main file :\verb' ../doc/Opisna.Rnw '
\subsection*{View as vignette}
Project files can be viewed by pasting this code to \R\ console:\\
\begin{Schunk}
\begin{Sinput}
> projectName <-"rps";  mainFile <-"Opisna"
\end{Sinput}
\end{Schunk}
\begin{Schunk}
\begin{Sinput}
  commandArgs()
  library(tkWidgets)
  openPDF(file.path(dirname(getwd()), "doc", paste(mainFile, 
+     "PDF", sep = ".")))
  viewVignette("viewVignette", projectName, file.path("../doc", 
+     paste(mainFile, "Rnw", sep = ".")))
\end{Sinput}
\end{Schunk}

\vfill \hrule \vspace{3pt} \footnotesize{
%Revision \SVNId\hfill (c) A. Blejec%\input{../_COPYRIGHT.}
%\SVNRevision ~/~ \SVNDate
\noindent
\texttt{Git Revision: \gitCommitterUnixDate \gitAbbrevHash{} (\gitCommitterDate)} \hfill \copyright A. Blejec\\
\texttt{ \gitReferences} \hfill \verb'../doc/Opisna.Rnw'\\

}



\end{document}
% ----------------------------------------------------------------
