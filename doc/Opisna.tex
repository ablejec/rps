% -*- TeX:Rnw -*-
% ----------------------------------------------------------------
% .R Sweave file  ************************************************
% ----------------------------------------------------------------
%%
% \VignetteIndexEntry{}
% \VignetteDepends{}
% \VignettePackage{}
%\documentclass[a4paper,12pt]{article}
\usepackage[slovene]{babel}
\usepackage[utf8]{inputenc} %% must be here for Sweave encoding check
\newcommand{\SVNRevision}{$ $Rev: 3 $ $}
%\newcommand{\SVNDate}{$ $Date:: 2009-02-2#$ $}
\newcommand{\SVNId}{$ $Id: program.Rnw 3 2009-02-22 17:36:08Z ABlejec $ $}
%\usepackage{babel}
%\input{abpkgB}
%\input{abpkg}
\input{abBeam}
\input{abcmd}
%\input{abpage}
\usepackage{pgf,pgfarrows,pgfnodes,pgfautomata,pgfheaps,pgfshade}
\usepackage{amsmath,amssymb}
\usepackage{colortbl}
\usepackage{Sweave}
\input{mysweaveB}
\newcommand{\BV}{}
\newcommand{\EV}{}
\newcommand{\myemph}[1]{{\color{Sgreen} \textit{#1}}}
\input{./figs/SwPres-concordance}
%\SweaveOpts{echo=false}
%\usepackage{lmodern}
%\input{abfont}

% ----------------------------------------------------------------
\title{RPS\\Analiza podatkov}
\author{A. Blejec}
%\address{}%
%\email{}%
%
%\thanks{}%
%\subjclass{}%
%\keywords{}%

%\date{}%
%\dedicatory{}%
%\commby{}%
\begin{document}
\mode<article> {\maketitle}
\mode<presentation> {\frame{\titlepage}}
\tableofcontents
% ----------------------------------------------------------------
\begin{abstract}
 Primer analize podatkov
\end{abstract}
% -------------------------------------------------------------
%% Sweave settings for includegraphics default plot size (Sweave default is 0.8)
%% notice this must be after begin{document}
% \setkeys{Gin}{width=0.9\textwidth}
\setkeys{Gin}{width=0.7\textwidth}
% ----------------------------------------------------------------
O rasti in velikosti ljudi imamo nekaj mnenj, ki jih lahko izrazimo v obliki raziskovalnih vprašanj. Najprej si zastavimo vprašanja.
%% |--------------------->>>>>>
\begin{frame}[fragile]
\frametitle{Vprašanja}
Nekaj vprašanj, na katere bi radi odgovorili je:
\begin{itemize}
  \item Ali so fantje večji od deklet?
  \item Ali so fantje težji od deklet?
  \item Ali sta razpon rok in višina približno enaka?
  \item Ali drži Galtonovo opažanje glede višine otrok in staršev?
  \item ...
\end{itemize}
Zbrali smo nekaj podatkov o študentih, s katerimi si bomo lahko poskusili odgovoriti.
\end{frame}
%% <<<<<<---------------------|
Nato zberemo podatke, s katerimi bomo poskusili odgovoriti na vprašanja. Ker predvidevamo, da nas bo zanimalo še kaj, zberemo podatke o še nekaj spremenljivkah.
\begin{Schunk}
\begin{Sinput}
> lfn <- "Podatki2012.txt"
\end{Sinput}
\end{Schunk}

%% |--------------------->>>>>>
\begin{frame}[fragile]
\frametitle{Podatki}
Podaki so o študentih 3. letnika biologije v letu 2012/13 so v datoteki \file{lfn}.
\begin{Schunk}
\begin{Sinput}
> fpath <- file.path("../data", lfn)
> data <- read.table(fpath, header = TRUE, sep = "\t")
> names(data)
\end{Sinput}
\begin{Soutput}
 [1] "starost" "mesec"   "spol"    "masa"    "visina" 
 [6] "roke"    "cevelj"  "lasje"   "oci"     "mati"   
[11] "oce"     "majica" 
\end{Soutput}
\end{Schunk}
\end{frame}
%% <<<<<<---------------------|

%% |--------------------->>>>>>
\begin{frame}[fragile]
\frametitle{Opisna statistika}
\begin{Schunk}
\begin{Sinput}
> summary(data[, 1:6])
\end{Sinput}
\begin{Soutput}
    starost          mesec        spol        masa      
 Min.   :20.00   Min.   : 0.000   F:33   Min.   :50.00  
 1st Qu.:21.00   1st Qu.: 5.000   M:10   1st Qu.:55.50  
 Median :21.00   Median : 7.000          Median :61.00  
 Mean   :22.07   Mean   : 6.814          Mean   :63.42  
 3rd Qu.:22.00   3rd Qu.: 9.500          3rd Qu.:70.00  
 Max.   :59.00   Max.   :11.000          Max.   :91.00  
                                                        
     visina           roke      
 Min.   :156.0   Min.   :154.0  
 1st Qu.:164.0   1st Qu.:163.2  
 Median :170.0   Median :167.8  
 Mean   :169.9   Mean   :169.3  
 3rd Qu.:173.5   3rd Qu.:172.5  
 Max.   :189.0   Max.   :193.0  
                 NA's   :5      
\end{Soutput}
\end{Schunk}
Ali kaj opazite?
\end{frame}
%% <<<<<<---------------------|
%% |--------------------->>>>>>
\begin{frame}[fragile]
\frametitle{Nenavadni podatki}
Kaj storiti s tistim, ki je napisal, da je rojen v mesecu 0?

Eden pa je star 59 let??
\end{frame}
%% <<<<<<---------------------|


%% |--------------------->>>>>>
\begin{frame}[fragile]
\frametitle{Nadaljevanje opisa}
\begin{Schunk}
\begin{Sinput}
> summary(data[, 7:dim(data)[2]])
\end{Sinput}
\begin{Soutput}
     cevelj      lasje  oci         mati      
 Min.   :36.00   S:19   S:24   Min.   :155.0  
 1st Qu.:38.00   T:24   T:19   1st Qu.:160.0  
 Median :39.00                 Median :165.0  
 Mean   :40.02                 Mean   :165.4  
 3rd Qu.:41.50                 3rd Qu.:168.0  
 Max.   :48.00                 Max.   :180.0  
                               NA's   :5      
      oce        majica 
 Min.   :170.0   L : 5  
 1st Qu.:174.2   M :19  
 Median :179.5   S :16  
 Mean   :179.1   XL: 1  
 3rd Qu.:182.0   XS: 2  
 Max.   :190.0          
 NA's   :5              
\end{Soutput}
\end{Schunk}
\end{frame}
%% <<<<<<---------------------|

\section{Višina in spol}

Primerjajte razpone vrednosti višin študentov in staršev.

%% |--------------------->>>>>>
\begin{frame}[fragile]
\frametitle{Višina po spolu}
Povzetek višin glede na spol
\begin{Schunk}
\begin{Sinput}
> summary(data$mati)
\end{Sinput}
\begin{Soutput}
   Min. 1st Qu.  Median    Mean 3rd Qu.    Max.    NA's 
  155.0   160.0   165.0   165.4   168.0   180.0       5 
\end{Soutput}
\begin{Sinput}
> by(data$visina, data$spol, summary)
\end{Sinput}
\begin{Soutput}
data$spol: F
   Min. 1st Qu.  Median    Mean 3rd Qu.    Max. 
  156.0   163.0   168.0   166.8   170.0   178.0 
--------------------------------------------- 
data$spol: M
   Min. 1st Qu.  Median    Mean 3rd Qu.    Max. 
  171.0   178.5   180.0   180.0   182.5   189.0 
\end{Soutput}
\begin{Sinput}
> summary(data$oce)
\end{Sinput}
\begin{Soutput}
   Min. 1st Qu.  Median    Mean 3rd Qu.    Max.    NA's 
  170.0   174.2   179.5   179.1   182.0   190.0       5 
\end{Soutput}
\end{Schunk}
\end{frame}
%% <<<<<<---------------------|

\section{Galton in višina otrok in staršev}
%% |--------------------->>>>>>
\begin{frame}[fragile]
\frametitle{Moški}
\begin{Schunk}
\begin{Sinput}
> with(data, plot(oce, visina, col = spol, pch = 16))
> abline(c(0, 1))
> abline(lm(visina ~ oce, data = data), col = 3, 
+     lwd = 3)
> abline(lm(visina ~ oce, data = data[data$spol == 
+     "M", ]), col = "red", lwd = 3)
> abline(lm(visina ~ oce, data = data[data$spol == 
+     "F", ]), col = "blue", lwd = 3)
\end{Sinput}
\end{Schunk}
\end{frame}
%% <<<<<<---------------------|

%% |--------------------->>>>>>
\begin{frame}[fragile]
\frametitle{Moški}
\includegraphics{./figs/SwPres-moski}
\end{frame}
%% <<<<<<---------------------|

%% |--------------------->>>>>>
\begin{frame}[fragile]
\frametitle{Ženske}
\begin{Schunk}
\begin{Sinput}
> with(data, plot(mati, visina, col = spol, pch = 16))
> abline(c(0, 1))
> abline(lm(visina ~ mati, data = data), col = 3, 
+     lwd = 3)
> abline(lm(visina ~ mati, data = data[data$spol == 
+     "M", ]), col = "red", lwd = 3)
> abline(lm(visina ~ mati, data = data[data$spol == 
+     "F", ]), col = "blue", lwd = 3)
\end{Sinput}
\end{Schunk}
\end{frame}
%% <<<<<<---------------------|

%% |--------------------->>>>>>
\begin{frame}[fragile]
\frametitle{Ženske}
\includegraphics{./figs/SwPres-zenske}
\end{frame}
%% <<<<<<---------------------|

%% |--------------------->>>>>>
\begin{frame}[fragile]
\frametitle{Moški}
\includegraphics{./figs/SwPres-moski2}
\end{frame}
%% <<<<<<---------------------|




% ----------------------------------------------------------------
%\bibliographystyle{amsplain}
%\bibliography{ab-general}
\clearpage
\section*{SessionInfo}
{\small
Windows 7 x64 (build 7601) Service Pack 1 \begin{itemize}\raggedright
  \item R version 2.15.1 (2012-06-22), \verb|x86_64-pc-mingw32|
  \item Locale: \verb|LC_COLLATE=Slovenian_Slovenia.1250|, \verb|LC_CTYPE=Slovenian_Slovenia.1250|, \verb|LC_MONETARY=Slovenian_Slovenia.1250|, \verb|LC_NUMERIC=C|, \verb|LC_TIME=Slovenian_Slovenia.1250|
  \item Base packages: base, datasets, graphics,
    grDevices, stats, utils
  \item Other packages: patchDVI~1.9
  \item Loaded via a namespace (and not attached):
    tools~2.15.1
\end{itemize}Project path:\verb' D:/_Y/R/rps '\\Main file :\verb' ../doc/Opisna.Rnw '
\subsection*{View as vignette}
Project files can be viewed by pasting this code to \R\ console:\\
\begin{Schunk}
\begin{Sinput}
> projectName <-"rps";  mainFile <-"Opisna"
\end{Sinput}
\end{Schunk}
\begin{Schunk}
\begin{Sinput}
> commandArgs()
> library(tkWidgets)
> openPDF(file.path(dirname(getwd()), "doc", paste(mainFile, 
+     "PDF", sep = ".")))
> viewVignette("viewVignette", projectName, file.path("../doc", 
+     paste(mainFile, "Rnw", sep = ".")))
\end{Sinput}
\end{Schunk}

\vfill \hrule \vspace{3pt} \footnotesize{
%Revision \SVNId\hfill (c) A. Blejec%\input{../_COPYRIGHT.}
%\SVNRevision ~/~ \SVNDate
\noindent
\texttt{Git Revision: \gitCommitterUnixDate \gitAbbrevHash{} (\gitCommitterDate)} \hfill \copyright A. Blejec\\
\texttt{ \gitReferences} \hfill \verb'../doc/Opisna.Rnw'\\

}

\end{document}
% ----------------------------------------------------------------
